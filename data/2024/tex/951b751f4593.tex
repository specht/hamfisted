\documentclass{article}
\usepackage[paperwidth=11cm]{geometry}
\usepackage[utf8]{inputenc}
\usepackage[german]{babel}
\usepackage[bitstream-charter]{mathdesign}
\let\circledS\undefined
\usepackage{amsmath,amssymb,amsfonts,amsthm}
\usepackage{tikz}
\usepackage{setspace}
\setstretch{1.15}
\pagestyle{empty}
\begin{document}
\setlength{\parindent}{0pt}
\begin{tikzpicture}
\draw [line width=0.01pt, opacity=0.01] (0,0) -- (\textwidth,0);
\end{tikzpicture}

\textbf{AC111}~~~An einem Kondensator mit einer Kapazität von 1 μF wird ein NF-Signal mit 10 kHz und 12 V$_{\textrm{eff}}$ angelegt. Wie groß ist die aufgenommene Wirkleistung im eingeschwungenen Zustand?

\vspace*{-2mm}
\begin{tikzpicture}
\draw [line width=0.01pt, opacity=0.01] (0,0) -- (\textwidth,0);
\end{tikzpicture}
\end{document}
