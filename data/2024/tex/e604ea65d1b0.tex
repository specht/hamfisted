\documentclass{article}
\usepackage[paperwidth=11cm]{geometry}
\usepackage[utf8]{inputenc}
\usepackage[german]{babel}
\usepackage[bitstream-charter]{mathdesign}
\let\circledS\undefined
\usepackage{amsmath,amssymb,amsfonts,amsthm}
\usepackage{tikz}
\usepackage{setspace}
\setstretch{1.15}
\pagestyle{empty}
\begin{document}
\setlength{\parindent}{0pt}
\begin{tikzpicture}
\draw [line width=0.01pt, opacity=0.01] (0,0) -- (\textwidth,0);
\end{tikzpicture}

\textbf{EC306}~~~Vorausgesetzt sind zwei Spulen in gleicher Umgebung, mit gleicher Windungszahl und mit gleicher Querschnittsfläche. Die erste Spule hat eine Induktivität von 12 μH. Die zweite Spule hat die doppelte Länge der ersten Spule. Wie hoch ist die Induktivität der zweiten Spule?

\vspace*{-2mm}
\begin{tikzpicture}
\draw [line width=0.01pt, opacity=0.01] (0,0) -- (\textwidth,0);
\end{tikzpicture}
\end{document}
